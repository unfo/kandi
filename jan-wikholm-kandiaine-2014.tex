% --- Template for thesis / report with tktltiki2 class ---
% 
% last updated 2013/02/15 for tkltiki2 v1.02

\documentclass[finnish]{tktltiki2}

% tktltiki2 automatically loads babel, so you can simply
% give the language parameter (e.g. finnish, swedish, english, british) as
% a parameter for the class: \documentclass[finnish]{tktltiki2}.
% The information on title and abstract is generated automatically depending on
% the language, see below if you need to change any of these manually.
% 
% Class options:
% - grading                 -- Print labels for grading information on the front page.
% - disablelastpagecounter  -- Disables the automatic generation of page number information
%                              in the abstract. See also \numberofpagesinformation{} command below.
%
% The class also respects the following options of article class:
%   10pt, 11pt, 12pt, final, draft, oneside, twoside,
%   openright, openany, onecolumn, twocolumn, leqno, fleqn
%
% The default font size is 11pt. The paper size used is A4, other sizes are not supported.
%
% rubber: module pdftex

% --- General packages ---

\usepackage[utf8]{inputenc}
\usepackage[T1]{fontenc}
\usepackage{lmodern}
\usepackage{microtype}
\usepackage{amsfonts,amsmath,amssymb,amsthm,booktabs,color,enumitem,graphicx}
\usepackage[pdftex,hidelinks]{hyperref}

% Automatically set the PDF metadata fields
\makeatletter
\AtBeginDocument{\hypersetup{pdftitle = {\@title}, pdfauthor = {\@author}}}
\makeatother

% --- Language-related settings ---
%
% these should be modified according to your language

% babelbib for non-english bibliography using bibtex
\usepackage[fixlanguage]{babelbib}
\selectbiblanguage{finnish}

% add bibliography to the table of contents
\usepackage[nottoc]{tocbibind}
% tocbibind renames the bibliography, use the following to change it back
\settocbibname{Lähteet}

% --- Theorem environment definitions ---

\newtheorem{lau}{Lause}
\newtheorem{lem}[lau]{Lemma}
\newtheorem{kor}[lau]{Korollaari}

\theoremstyle{definition}
\newtheorem{maar}[lau]{Määritelmä}
\newtheorem{ong}{Ongelma}
\newtheorem{alg}[lau]{Algoritmi}
\newtheorem{esim}[lau]{Esimerkki}

\theoremstyle{remark}
\newtheorem*{huom}{Huomautus}


% --- tktltiki2 options ---
%
% The following commands define the information used to generate title and
% abstract pages. The following entries should be always specified:

\title{Madonreiät langattomissa ad hoc -verkoissa}
\author{Jan Wikholm}
\date{\today}
\level{Kandidaatintutkielman aineversio}
\abstract{Madonreikä-hyökkäysten ja niiden vastatoimien tyypitys.}

% The following can be used to specify keywords and classification of the paper:

\keywords{ad hoc -verkot, wlan, hyökkäys, puolustus, havainnointi}

% classification according to ACM Computing Classification System (http://www.acm.org/about/class/)
% This is probably mostly relevant for computer scientists
% uncomment the following; contents of \classification will be printed under the abstract with a title
% "ACM Computing Classification System (CCS):"
% \classification{}

% If the automatic page number counting is not working as desired in your case,
% uncomment the following to manually set the number of pages displayed in the abstract page:
%
% \numberofpagesinformation{16 sivua + 10 sivua liitteissä}
%
% If you are not a computer scientist, you will want to uncomment the following by hand and specify
% your department, faculty and subject by hand:
%
% \faculty{Matemaattis-luonnontieteellinen}
% \department{Tietojenkäsittelytieteen laitos}
% \subject{Tietojenkäsittelytiede}
%
% If you are not from the University of Helsinki, then you will most likely want to set these also:
%
% \university{Helsingin Yliopisto}
% \universitylong{HELSINGIN YLIOPISTO --- HELSINGFORS UNIVERSITET --- UNIVERSITY OF HELSINKI} % displayed on the top of the abstract page
% \city{Helsinki}
%


\begin{document}

% --- Front matter ---

\frontmatter      % roman page numbering for front matter

\maketitle        % title page
\makeabstract     % abstract page

\tableofcontents  % table of contents

% --- Main matter ---

\mainmatter       % clear page, start arabic page numbering

\section{Johdanto}

% Write some science here.


Langattomat päätelaitteet -- kuten matkapuhelimet, PDA:t ja kannettavat tietokoneet -- voivat muodostaa langattoman ad hoc -verkon, jonka avulla ne voivat kommunikoida ilman erillistä verkkoinfrastruktuuria. \cite{delphi}. Sensori- ja ad hoc -verkot voivat toimia viestintäalustana monissa erilaisissa käyttötarkoituksissa kuten pelastus-, armeija- \cite{leashes} sekä myös siviilikäytössä \cite{liteworp}. Esimerkiksi luonnonkatastrofin jäljiltä perinteiset langattomat tukiasemat voivat olla tuhoutuneet ja pelastuslaitosten työntekijät voivat olla viestinnässään ad hoc -verkkojen varassa \cite{leashes}.

\noindent \\
Näiden verkkojen suurimpia etuja ovat käyttöönoton nopeus ja kustannustehokkuus \cite{delphi,leashes}, sillä laitteisto on usein edullista ja päätelaitteet osaavat itsenäisesti luoda verkon. Vaikka ad hoc -verkkoja voi muodostaa myös langallisesti, on useimmiten käytössä langattomat teknologiat \cite{leashes} ja siksi keskitymme niihin.

\noindent \\
Pääosa teknologian alkuvaiheen tutkimuksesta on keskittynyt näiden lupausten toteuttamiseen luoden reititysprotokollia ja muita välttämättömiä viestinnän osia \cite{liteworp}. Ad hoc -verkkojen avoimuuden ja autonomisuuden seurauksena ne ovat erityisen haavoittuvia monille erilaisille hyökkäyksille: \emph{salakuuntelu} (eavesdropping), \emph{väärennys} (spoofing) ja \emph{toistaminen} (replay) \cite{leashes}. Näiden lisäksi hyökkääjä voi tahallisesti olla välittämättä paketteja,  \emph{musta aukko -hyökkäys} (blackhole attack), tai syöttää niitä verkkoon paljon tukehduttaakseen sen järkevän käytön,  \emph{valkoinen aukko -hyökkäys} (white hole attack) \cite{delphi}. \emph{Madonreikähyökkäys} (wormhole attack) on erityisen vakava hyökkäys ad hoc -verkoissa \cite{liteworp}.

\noindent \\
Madonreikähyökkäyksessä kaksi tai useampi paha-aikeista tahoa toimivat yhteistyössä saadakseen liikenteen ohjautumaan niiden välillä kulkevaa reittiä pitkin, jotta voivat toteuttaa edellä mainittuja hyökkäyksiä. Nämä tahot välittävät kaikki kuulemansa paketit toiselle osapuolelle, joka toistaa ne omassa päässään. Tämä pakettien välitys voidaan toteuttaa dedikoidulla suurinopeuksisella linkillä, pakettien kapseloinnilla normaalia verkkoa pitkin tai vaikka suuritehoisella lähettimellä. \cite{liteworp}. Tunnelin ollessa toiminnassa se häiritsee reititysprotokollia tarjoten lyhimmän ja yleensä nopeimman reitin, joten muut verkon laitteet päätyvät lähettämään suuren osan paketeista sen läpi. Erityisen salakavalan hyökkäyksestä tekee se, että hyökkääjien ei tarvitse murtaa mitään salausta koska koko hyökkäys perustuu pakettien kopiointiin (salakuunteluun ja sen jälkeiseen toistoon) verkon osasta toiseen.

\noindent \\
Esittelemme luvussa 2 madonreikähyökkäysten hyökkääjä- \cite{delphi} sekä hyökkäystyypit \cite{liteworp} minkä jälkeen kerron laitteistoriippuvaisista puolustuskeinoista luvussa 3 ja protokollapohjaisista puolustusmekanismeista luvussa 4. Yhteenvedon näistä esitämme luvussa 5.

\section{Hyökkääjä- ja hyökkäystyypit}

Madonreikähyökkääjiä on kahta eri tyyppiä: \emph{piilotettu} ja \emph{avoin} \cite{delphi} ja hyökkäyksiä on viittä eri tyyppiä. \cite{liteworp}. Kaikkia hyökkäystyyppejä ei ole käsitelty kaikissa puolustuskeinoissa, joten käsittelemättömien hyökkäystyyppien torjumisen toimivuus on erinomainen kohde jatkotutkimukselle.

\subsection{Piilotettu hyökkääjä}

Piilotetut hyökkääjät toimivat verkossa kertomatta muille verkon laitteille omasta olemassaolostaan. Ne kuuntelevat liikennettä ja siirtävät paketteja madonreiän läpi täysin muokkaamatta. Tällöin kaukanakin olevat laitteet voivat luulla madonreiän läpi tulevia paketteja naapureilta tuleviksi, koska eivät tiedä välissä olevan toistimena toimiva madonreikä.

Esimerkiksi pakettihihnat toimivat piilotettuja hyökkääjiä vastaan.

\subsection{Avoin hyökkääjä}

Avoimet hyökkääjät rekisteröityvät verkkoon kuten muutkin laitteet. Muille verkon laitteille näyttää siltä, että nämä laitteet ovat ensimmäisen asteen naapureita ja siten niiden kautta löytyvä lyhyt polku on täysin käypä vaihtoehto. Tapa, jolla madonreikä on muodostettu, on jokin alla kuvailluista viidestä hyökkäystavasta.

\subsection{Pakettikapselointi}

Pakettikapseloinnissa hyökkääjät H1 ja H2 voivat käyttää jo olemassa olevaa verkkoa: H1 kuulee paketin ja luo uuden H2:lle suunnatun paketin, jonka sisältönä on sellaisenaan H1:n kaappaama paketti. Kun tämä kapseloitu paketti saavuttaa H2:n se toistaa alkuperäisen sisällön sellaisenaan verkkoon, joten sen naapurit luulevat H1:n naapureiden olevan lähellä.

\subsection{Erilliskaistahyökkäys}

Mikäli hyökkääjillä on normaalin lähetyskaistan - esim. wlan-verkko - lisäksi käytössä vaikkapa yksinkertaisesti kytketty ethernetverkko, voivat ne kommunikoida yleistä verkkoa nopeammin ja sen kantamaa pidemmälle. Tätä kutsutaan erilliskaistahyökkäykseksi ja tämä on nimenomaan se hyökkäys, johon suurin osa puolustuskeinoista viittaa itse madonreikähyökkäyksenä.

\subsection{Suurteholähetys}

Yksi oletus, mikä puolustuskeinoja laatiessa pitää pitää mielessä on, että hyökkääjille pitää olettaa loputtomat resurssit toisin kuin muille verkon laitteille, joita nimenomaan yleensä yhdistää resurssien vähyys. Tästä oletuksesta yhtenä esimerkkinä on suurteholähetys: hyökkääjälaite lähettää kaappaamansa reitityspaketit huomattavan suurella lähetysteholla ja täten saa aikaan sen itsensä lävitse kulkevan lyhyimmän reitin. Tämä hyökkäys ei siis vaadi kahta osapuolta. 

\subsection{Pakettivälitys}

Kuten suurteholähetys pakettivälityshyökkäys ei vaadi hyökkääjäparia vaan sen voi suorittaa yksikin vihamielinen laite. Tässä hyökkääjä on kahden laitteen välissä välittäen paketteja jolloin nämä laitteet luulevat olevansa naapureita ja hyökkääjä niiden välissä voi suorittaa esimerkiksi salakuuntelua tai palvelunestohyökkäystä.


\subsection{Protokollapoikkeamat}

Protokollapoikkeamilla tarkoitetaan paha-aikeisten laitteiden tahallista verkkoprotokollan rikkomista: esimerkiksi pakettitörmäysten estämiseksi tietyt protokollat vaativat reitityspakettien lähetyksessä pientä viivettä - paha-aikeinen laite voi täten lähettää paketit heti ja aiheuttaa tavallisille laitteille haittaa törmäyksillä. Toinen vaihtoehto on reitityspakettien lähettämättä jättäminen jolloin verkon reititys ja polunetsintä ei toimi oikein. Kummassakin tapauksessa hyökkääjä voi junailla toimensa siten, että suuri osa verkon liikenteestä päätyy reitittymään sen läpi.

\section{Laitteistoriippuvaiset puolustusmekanismit}

Nämä puolustusmekanismit eivät vaadi reititysprotokolliin muutoksia, mutta niillä on laitteistovaatimuksia.

\subsection{Aika- ja geohihnat}
Yih-Chun Hu et al \cite{leashes} kuvailevat uuden mekanismin - \emph{pakettihihnat} (packet leashes) - ja sen kaksi eri varianttia: \emph{aikahihnan} (temporal leash) ja \emph{geohihnan} (geographic leash). Tässä hihnalla tarkoitetaan sellaista dataa, jolla paketin enimmäiskantamaa voidaan rajoittaa.

\noindent \\
Pakettihihnojen lisäksi he luovat uuden TIK-verkkoprotokollan, joka käyttää aikahihnoja madonreikiä vastaan.

\paragraph{Aikahihnat} 
\noindent \\
Aikahihnojen edellytyksenä on tarkka kellojen synkronointitarkkuus: muutaman mikrosekunnin tai jopa satojen nanosekuntien tarkkuus. Kaikkien laitteiden pitää myöskin olla tietoisia virhemarginaalin suuruusluokasta kahden laitteen välillä. Tällainen synkronointitarkkuus onnistuu esimerkiksi GPS:n avulla. Vaikka tarkkuusvaatimus on erittäin tiukka, on se kirjoittajien mukaan täysin hyväksyttävä ottaen huomioon madonreikähyökkäyksen vakavuuden.

\noindent \\
Aikahihnaa muodostaessa lähettäjä päättää enimmäispituuden lähetykselle ja laskee kauanko valonnopeudella kulkevalla radiosignaalilla kestää sinne päätyä – tässä tietenkin otetaan huomioon synkronoinnin virhemarginaali – ja tämän avulla lähettäjä laskee paketille vanhenemisajan, jonka jälkeen paketti on hylättävä. Tunnelointi aiheuttaa pakosti viivettä, koska se kulkee kauemmaksi, joten madonreiän toisella puolella olevat laitteet pudottavat vanhentuneet paketit.

\noindent \\
Vaikka aikahihna on hihnatyypeistä tarkempi on sen haasteena se, ettei lähettäjä tiedä aina tarkalleen omaa lähetysaikaansa, koska fyysisen kerroksen lähetysmekanismi voi joutua odottamaan; täten on vaikeaa etukäteen luoda lähetysajankohtaan perustuvaa digitaalista allekirjoitusta.

\paragraph{Geohihnat}
\noindent \\
Geohihnoja käytettäessä kaikkien verkon laitteiden pitää tietää oma sijaintitietonsa sekä niiden pitää pystyä synkronoimaan kellonsa muiden kanssa; joskin kellojen synkronointitarkkuus ei ole niin tärkeä seikka, koska laitteiden liikkumisvauhti suhteessa valonnopeuteen on marginaalinen.

\noindent \\
Geohihnan toiminta on hyvin yksinkertainen: laite tarkistaa onko sen oma sijainti alkuperäisessä paketissa määritellyn sallitun matkan päässä. Koska paketit allekirjoitetaan digitaalisesti, voi laite luottaa paikannustiedon olevan alkuperäiseltä lähteeltä. Toisin kuin pelkkä matkan pituuteen perustuva tarkistus geohihnat toimivat myös siinä tilanteessa, että madonreikä kuljettaa paketin jonkin esteen ohi.

\noindent \\
Geohihnoissa laitteet tietävät toistensa oletetun maksimiliikenopeuden ja täten madonreikätunneloitu paketti voidaan tunnistaa ilkivaltaisen laitteen lähettämäksi, jos se lähettää verkkoon kaksi pakettia joiden lokaatiotiedoissa on tapahtunut maksimiliikenopeutta nopeampaa liikettä edellyttävä muutos.

\paragraph{TIK-protokolla} 
\noindent \\
TIK-protokollan nimi tulee sanoista ``TESLA with Instant Key disclosure``, eli se on TESLA-protokollan laajennus välittömällä avainten paljastuksella. TIK-protokolla käyttää kirjoittajien aikahihnoja.

\noindent \\
Tärkeimpänä ominaisuutena koko aikahihnafunktion toiminnassa on aikaleimojen luotettavuus; leimat pitää pystyä todentamaan. Jaetut avaimet hylätään suoraan todeten niiden hallinan olevan liian raskas operaatio. Toisena ideana on digitaalisen allekirjoituksen liittäminen jokaiseen pakettiin, jolloin jokaiselle laitteelle riittää yksi julkinen-yksityinen–avainpari ja jokaisen laitteen tarvitsee tietää vain tämä kaikkien julkisten avainten joukko. Digitaaliset allekirjoitukset kuitenkin yleensä perustuvat raskaaseen asymmetriseen kryptografiaan, joka ei sovi ad hoc –verkkojen oletettuun resurssivähyyteen.

\noindent \\
Symmetriseksi vaihtoehdoksi raportti tarjoaa tiivisteistä koostuvaa binääripuuta (Merkle-tiivistepuu). Tiivisteiden hyväksi puoleksi kerrotaan niiden erittäin tehokas laskeminen ja yksisuuntaisuus. Merkle-puussa jokainen tiiviste muodostuu kahden lapsensa tiivisteistä aina juureen saakka. Tämä juuritiiviste on laitteen julkinen avain ja alimman tason lapset ovat yksityinen avain.

\noindent \\
Lähettäjän avainnippu koostuu satunnaisluvuista, jotka on tallennettu Merkle-puun pohjalle siten että ne on indeksoitu juoksevaan järjestykseen. Tätä järjestyslukua vastaa ajanhetkien järjestysluku; ajanlasku alkaa laitteiden yhtenään sopimasta hetkestä ja etenee sovituin lisäyksin (esimerkiksi 11,5 mikrosekuntia). 

\noindent \\
Vastaanottaja tietää jo etukäteen lähettäjän julkisen avaimen, eli Merkle-puun juuritiivisteen, ja jokaisen paketin yhteydessä se saa neljä uutta datapalaa:

\begin{enumerate}
\item tiivisteen viestistä ja avaimesta $K_i$,
\item viestin,
\item tiedot, joiden avulla avaimesta $K_i$ saadaan laskettua juuritiiviste, sekä
\item avaimen $K_i$.
\end{enumerate}

\noindent \\
Nämä osat tulevat nimenomaan tarkasti tässä järjestyksessä. Koska avainta ei ole viestin tiivistelaskennan aikaan vielä lähetetty, vastaanottaja voi nyt olla varma, ettei kukaan ole voinut väärentää tiivistettä. Vastaanottajan tarkistettua viestin aitouden se tarkistaa vielä aikahihnan rajat ja katsoo, onko paketti voimassa.

\paragraph{Resurssivaativuus} 
\noindent \\
Resurssien niukkuuteen kirjoittajat mainitsevat pääratkaisuna symmetrisen kryptografian käyttö asymmetrisen sijaan – operaatioiden nopeuserot voivat olla kolmesta neljään suuruusluokkaa (100–1000-kertaisia) hitaampia jälkimmäisessä.

\noindent \\
TIK-protokollaa arvioidessaan kirjoittajat tarkastelivat nykyisten (2001) mobiililaitteiden laskentatehoa ja muistimäärää. He toteavat protokollan laskenta- ja muistivaatimukset mahdollisiksi saatavilla olevilla laitteilla; joskin huomauttavat ettei TIK sovi resursseiltaan aivan kaikkein köyhimpiin laitteisiin kuten sensoriverkkoihin.

\noindent \\
Allekirjoitusta varten muodostettavan tiivistepuun muodostusta ja tallennusta voi optimoida siten että vain osa puusta säilötään muistissa ja osa lasketaan tarvittaessa. Tällaisen optimoinnin takia yhden vuorokauden yksityiset avaimet sisältävä osittainen puu saadaan mahtumaan 2,5 megatavuun kokonaisen puun 170 gigatavun sijaan.

\noindent \\
Kirjoittajien yhteenveto protokollastaan: TIK-protokolla suojaa laitteita tehokkaasti ilkivaltaiselta toistolta, väärennykseltä ja madonreikähyökkäyksiltä; sekä varmistaa tuoreuden. Protokolla on toteutettavissa nykylaitteistoilla vaikka se ei sovikaan kaikista rajoitetuimpiin sensoriverkkoihin.

\subsection{Suunta-antenni}
Lingxuan Hu ja David Evans kuvaavat suunta-antennin käyttöä madonreikien estämisessä \cite{antenna}
\begin{itemize}
\item laitteiden sisäisen kompassin tarkka suuntima
\item magneeteilla häiriötä
\item vaatii 3. osapuolen todentamaan liikenteen suuntaa
\item olettaa linkkien väliset salaukset
\item naapurilistat
\item Worawannotai-hyökkäys (erikoistapaus todentaja-aseman väärinkäytöstä)
\end{itemize}

\section{Puhtaasti protokollapohjaiset puolustusmekanismit}
Seuraavilla ratkaisuilla on laajempi käyttöpotentiaali, koska ne eivät vaadi erityislaitteistoa.

\subsection{DeWorm}

Hayajneh et al kuvailevat DeWorm-protokollan \cite{deworm}

\begin{itemize}
\item Isossa verkossa raskas
\item verkkoliikenne-kustannukset
\item pala palalta polun tarkistus jokaiselle eri polulle
\end{itemize}

\subsection{DelPHI}

Hon Sun Chiu ja King-Shan Lui kertovat viiveeseen perustuvasta DelPHI-protokollastaan \cite{delphi}

\begin{itemize}
\item Kokonaiskesto RTT / hyppyjen määrällä
\item Normaali verkko: A->B->C->D->E (4 hyppyä) 
\item Rei'itetty salattu verkko: A->(M1->M2)->E (1 hyppy)
\item rei'itetty on nopeampi, mutta RTT / hyppyjen määrällä on sillä selvästi isompi kuin pienin hypyin etenevä rehti verkko => madonreikä.
\end{itemize}

\subsection{LiteWorp}

Khalil et al esittelevät naapurilistoihin ja vartiointiin perustuvan LiteWorp-protokollan \cite{liteworp}

\begin{itemize}
\item vartijat
\item väärät syytökset vs verkkopakettien törmäilyt
\item vahtilistojen ja -puskurien tilavaativuudet
\item protokollan heikkeneminen tiheissä verkoissa (naapuri-lkm >20)
\end{itemize}

% --- References ---
%
% bibtex is used to generate the bibliography. The babplain style
% will generate numeric references (e.g. [1]) appropriate for theoretical
% computer science. If you need alphanumeric references (e.g [Tur90]), use
%
%
% instead.
\newpage

%\bibliographystyle{babplain-lf}
\bibliographystyle{babalpha-fl}
\bibliography{references-fi}


% --- Appendices ---

% uncomment the following

% \newpage
% \appendix
% 
% \section{Esimerkkiliite}

\end{document}
